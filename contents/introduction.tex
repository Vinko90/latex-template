\chapter{Introduction}
Short introduction, can be used to cite documents \cite{label1}.

\section{Background and context}

\section{Definitions and Abbreviations}

\subsection{Definitions}

\wodunderline{Text marking}

\begin{tabularx}{\textwidth}{p{2.5cm}X}
	\TODO{Marked text}       & Text needs to be changed or completed.\\
	\NEW{Marked text}        & Text has changed compared to the previous release.\\
	\MEASURE{Marked section} & Section headers that are intended for review.\\
\end{tabularx}%

% Remove the number notations when not used.
\wodunderline{Numbers}

\begin{tabularx}{\textwidth}{p{2.5cm}X}
	`\textit{a}'             & Numeric binary notation (\textit{a} can be multiple 0s or 1s). E.g. `\texttt{010}' is a 3-bit value representing the binary number two. This kind of notation implies a specific bit length.\\
	`\textit{aa.aaaa}'       & Numeric binary notation with `.' separations for clear reading of long binary numbers.\\
	0x\textit{a}             & Numeric hexadecimal notation (\textit{a} can be a digit 0 through 9, A through F). E.g. `\texttt{0x1A}' is hexadecimal number twenty-six. This kind of notation does not directly imply a bit length.\\
	0x\textit{aa.aaaa}       & Numeric hexadecimal notation with '.' separations for clear reading of long hexadecimal numbers.\\
	\textit{a}d              & Numeric (explicit) decimal notation. This kind of notation does not directly imply a bit length.\\
	X[b:a]                   & Vector notation for vector X with bit range b downto a (little endian notation).\\
\end{tabularx}%

\subsection{Abbreviations}
\begin{tabularx}{\textwidth}{lX}
	AFK   & Away from keyboard\\
	AFAIK & As far as I know\\
	BRB   & Be right back\\
\end{tabularx}%
